\section{Struttua Data Base}

Definiamo quindi ora tutte le tabelle con i rispettivi campi nel dettaglio 
\begin{table}[ht]
\centering
\caption{Tabella Docente}
\begin{tabularx}{\textwidth}{lX}
\toprule
\textbf{Campo} & \textbf{Descrizione} \\
\midrule
email & VARCHAR(255) - Chiave primaria \\
pass & VARCHAR(255) - Non nullo \\
nome & VARCHAR(255) - Non nullo \\
cognome & VARCHAR(255) - Non nullo \\
dataDiNascita & DATE \\
\bottomrule
\end{tabularx}
\label{tab:docente}
\end{table}

\begin{table}[ht]
\centering
\caption{Tabella Segreteria}
\begin{tabularx}{\textwidth}{lX}
\toprule
\textbf{Campo} & \textbf{Descrizione} \\
\midrule
email & VARCHAR(255) - Chiave primaria \\
pass & VARCHAR(255) - Non nullo \\
root & boolean - Non nullo \\

\bottomrule
\label{tab:segreteria}

\end{tabularx}
\end{table}


\begin{table}[ht]
\centering
\caption{Tabella CorsoDiLaurea}
\begin{tabularx}{\textwidth}{lX}
\toprule
\textbf{Campo} & \textbf{Descrizione} \\
\midrule
id & SERIAL - Chiave primaria \\
nome & VARCHAR(255) - Non nullo \\
durata & INTEGER - Non nullo \\
anno & VARCHAR(255) - Non nullo \\
desc & text - Non nullo \\

\bottomrule
\label{tab:corsoDiLaurea}

\end{tabularx}
\end{table}


\begin{table}[ht]
\centering
\caption{Tabella Insegnamento}
\begin{tabularx}{\textwidth}{lX}
\toprule
\textbf{Campo} & \textbf{Descrizione} \\
\midrule
id & SERIAL - Chiave primaria \\
nomeInsegnamento & VARCHAR(255) \\
annoConsigliato & INTEGER - Non nullo \\
cfu & INTEGER \\
desc & TEXT\\
corsoDiAppartenenza & INTEGER - Chiave esterna, riferisce a "corsoDiLaurea" ("id") - Non nullo \\

responsabile & VARCHAR(255) - Chiave esterna, riferisce a "docente" ("email") \\

\bottomrule
\label{tab:insegnamento}

\end{tabularx}
\end{table}

\begin{table}[ht]
\centering
\caption{Tabella Propedeuticità}
\begin{tabularx}{\textwidth}{lX}
\toprule
\textbf{Campo} & \textbf{Descrizione} \\
\midrule
id\_insegnamento & INTEGER - Chiave esterna, riferisce a "insegnamento" ("id") \\
id\_insegnamento\_propedeutico & INTEGER - Chiave esterna, riferisce a "insegnamento" ("id") \\
\bottomrule
\label{tab:propedeuticità}

\end{tabularx}
\end{table}


\begin{table}[ht]
\centering
\caption{Tabella Studente}
\begin{tabularx}{\textwidth}{lX}
\toprule
\textbf{Campo} & \textbf{Descrizione} \\
\midrule
matricola & VARCHAR(6) - Chiave primaria \\
email & VARCHAR(255) - Unico, non nullo \\
pass & VARCHAR(255) - Non nullo \\
nome & VARCHAR(255) - Non nullo \\
cognome & VARCHAR(255) - Non nullo \\
cfu & INTEGER - Default 0 \\
idLaurea & INTEGER - Chiave esterna, riferisce a "corsoDiLaurea" ("id") \\
dataN & DATE - Non nullo \\
\bottomrule
\label{tab:studente}

\end{tabularx}
\end{table}



\begin{table}[ht]
\centering
\caption{Tabella Appello}
\begin{tabularx}{\textwidth}{lX}
\toprule
\textbf{Campo} & \textbf{Descrizione} \\
\midrule
dataA & TIMESTAMP \\
luogo & VARCHAR(255) \\
corso & INTEGER - Chiave esterna, riferisce a "insegnamento" ("id") \\
\bottomrule
\label{tab:appello}

\end{tabularx}
\end{table}

\begin{table}[ht]
\centering
\caption{Tabella Sostiene}
\begin{tabularx}{\textwidth}{lX}
\toprule
\textbf{Campo} & \textbf{Descrizione} \\
\midrule
id_corso & INTEGER \\
data & TIMESTAMP \\
id_studente & VARCHAR(6) - Chiave esterna, riferisce a "Studente" ("matricola") \\
voto & voto \\
\bottomrule
\label{tab:sostiene}
\end{tabularx}
\end{table}


\begin{table}[ht]
\centering
\caption{Tabella Studente\_arc}
\begin{tabularx}{\textwidth}{lX}
\toprule
\textbf{Campo} & \textbf{Descrizione} \\
\midrule
matricola & VARCHAR(6) - Chiave primaria \\
email & VARCHAR(255) - Unico e non nullo \\
pass & VARCHAR(255) - Non nullo \\
nome & VARCHAR(255) - Non nullo \\
cognome & VARCHAR(255) - Non nullo \\
cfu & INTEGER - Predefinito a 0 \\
periodoInattivita & INTERVAL \\
idLaurea & INTEGER - Chiave esterna, riferisce a "corsoDiLaurea" ("id") \\
dataN & DATE - Non nullo \\
\bottomrule
\label{tab:studenteArc}
\end{tabularx}
\end{table}

\begin{table}[ht]
\centering
\caption{Tabella Voti\_arc}
\begin{tabularx}{\textwidth}{lX}
\toprule
\textbf{Campo} & \textbf{Descrizione} \\
\midrule
id & SERIAL - Chiave primaria \\
voto & voto - Non nullo \\
dataEsame & DATE \\
studente & VARCHAR(6) - Chiave esterna, riferisce a "Studente\_arc" ("matricola") \\
\bottomrule
\label{tab:votiArc}

\end{tabularx}
\end{table}

\begin{table}[ht]
\centering
\caption{Tabella Insegnamento\_arc}
\begin{tabularx}{\textwidth}{lX}
\toprule
\textbf{Campo} & \textbf{Descrizione} \\
\midrule
id\_voto & INTEGER - Chiave esterna, riferisce a "Voti\_arc" ("id") \\
id\_insegnamento & INTEGER - Chiave esterna, riferisce a "insegnamento" ("id") \\
\bottomrule
\label{tab:insegnamentoArc}

\end{tabularx}
\end{table}